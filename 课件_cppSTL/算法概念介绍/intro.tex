\documentclass{beamer}
\usepackage{xeCJK}
\useinnertheme{rounded}
\usecolortheme{seahorse}
\useoutertheme{smoothbars}
\begin{document}
\title{香农先修班第一次课——见面会}
\subtitle{算法导学}
\institute{软件协会香农先修班}
\author{lr580}
\date{2022年7月13日}
\logo{\includegraphics[width=70pt]{logo.jpg}}
\begin{frame}
    \titlepage
\end{frame}

% \AtBeginSection[]{
\begin{frame}
    \frametitle{目录}
    \tableofcontents%[currentsection]
\end{frame}
% }

\section{香农先修班介绍}
\begin{frame}
    \begin{block}{先(xiu)修(xian)班简介}
        香农先修班是软件协会下属组织,旨在传授入门和基础的算法知识,这些知识在算法竞赛与企业面试中会有较大帮助。
    \end{block}
    {\large\textbf{\color[RGB]{23, 130, 20}先修班负责人介绍}}\\
    {\footnotesize 注:先按年级排序,后按拼音排序。}\\
    \begin{itemize}
        \item (20级)覃梓鑫
        \item (21级)陈嘉成、陈家骏、李勉鑫、袁子锋
    \end{itemize}
    {\footnotesize (21级负责人是先修班讲课的主要人员。)}\\~\\
    授课内容与时间参见 SCNUOJ 小组。
\end{frame}

\newtheorem{chdefinition}{定义}
\section{算法的概念与学习指导}
\subsection{算法的定义}
\begin{frame}
    \begin{chdefinition}
        \textbf{算法}\ (algorithm)\\
        {\footnotesize 注:这里特指传统算法(区分于人工智能算法),而且是狭义的算法(广义的传统算法是一切满足有穷性、确切性等的程序执行步骤,即 $a+b$ 也是算法)。}\\
        具体而言,在先修班里,我们默认算法就是具体算法知识点的总和。这些知识点包括搜索、动态规划、数据结构、图论、数论等。
    \end{chdefinition}
    {\footnotesize 这些具体算法知识点可参见“XCPC知识树”。}\\~\\
    \large{\textbf{\color[RGB]{250, 148, 27}算法的作用}}\\
    \begin{itemize}
        \item 求解问题的答案
        \item 优化求解问题的用时/占用内存
    \end{itemize}
\end{frame}

\subsection{学习算法的好处与条件}
\begin{frame}
    \begin{block}{学习算法的好处}
        \begin{itemize}
            \item 提升代码基本功,写项目的代码逻辑、debug等会比较轻松
            \item 以后学算法与数据结构专业课易如反掌
            \item 公司面试的算法部分能较轻松应对
            \item 参加算法竞赛有机会获得省奖、国家奖等
            \item 提升个人能力,如锻炼思维
        \end{itemize}
    \end{block}
    \pause
    \large{\textbf{\color[RGB]{218, 76, 21}学习算法的前置条件}}\\
    \begin{enumerate}
        \item \textbf{充足的时间精力}
        \item 掌握一门编程语言
        \item 一定的数学与英语基础
    \end{enumerate}
\end{frame}

\subsection{学习算法的路线指导}
\begin{frame}
    \begin{block}{基本学习方法}
        \begin{itemize}
            \item 以做题练习为主
            \item 重理解,而不是重记背
        \end{itemize}
    \end{block}
    \large{\textbf{\color[RGB]{218, 76, 21}注意事项}}\\
    \begin{itemize}
        \item 算法学习周期较长,一般来说要数月努力才能看到明显的进步(特别是后期)
        \item 多交流,学习他人代码和解题思路的优点
        \item 看得懂与会做是两回事(重要的是如何想出来的)
    \end{itemize}
\end{frame}

\subsection{相关资源}
\begin{frame}
    \large{\textbf{\color[RGB]{34, 164, 241}资源网站}}\\
    \begin{enumerate}
        \item oi-wiki\par %](https://oi-wiki.org/)
        特点:算法的百科全书,较全的知识库,适合查资料,不适合顺序跟随学习。
        \item 算法竞赛入门到进阶(罗勇军,郭卫斌)\par
        特点:涵盖面广,但难度可能略高。
        \item 挑战程序设计竞赛2\ 算法和数据结构(渡步有隆)\par %](https://cloud.socoding.cn/s/a8Fe?password=blue)
        特点:较适合初学者入门,但知识并不是非常全。
        \item 网络上海量的学习笔记、题解等\par
        特点:散乱,质量参差不齐。\\但能补充知识,适合专题学习。
    \end{enumerate}
\end{frame}

\begin{frame}
    \large{\textbf{\color[RGB]{34, 164, 241}做题网站(OJ)}}\\
    \begin{enumerate}
        \item 洛谷\\
        特点:较多优秀的官方与非官方题单,题解丰富。
        \item Codeforces\\
        特点:有很多比赛真题。也适合打比赛练手(就是时间有点阴间),比赛偏思维。
        \item vjudge\\
        特点:是 OJ 的集合,收集了各 OJ 的题。\\缺点是不稳定。
    \end{enumerate}
\end{frame}

\subsection{算法学习的水平}
\begin{frame}
    通常而言,算法能力水平能粗略的划分如下:
    \begin{enumerate}
        \item \textbf{入门}(非先修班多数人的最终水平)\\
        效果:达到专业课要求,能解决算法简单题和应付面试。\\
        能力:能做出洛谷大部分橙题。
        \item \textbf{基础}(先修班主体的平均水平)\\
        效果:能获得省级或国家级算法比赛的多数奖项。\\
        能力:洛谷全部官方题单掌握约 70\%,橙黄模板基本都会。
        \item \textbf{进阶}(一个年级通常十数人或几人)\\
        效果:能获 ACM 竞赛区域赛铜牌或以上。\\
        表现:至少能通过软院集训队选拔赛进入集训队。
    \end{enumerate}
\end{frame}

\section{主流算法比赛简介}
\begin{frame}
    \begin{alertblock}{注:}
        \begin{itemize}
            \item 所有比赛都能通过官网的或其他途径了解到详细信息。\\
            因此,强烈建议有空自行去翻翻各比赛官网。
            \item 这里介绍的比赛有:蓝桥杯、天梯赛、CCF、ACM。\\
            此外还有不少算法比赛,如码加加、赛氪(四季赛)等也是国家级比赛,比较好混综测加分,可自行了解,不赘述。\\
            还有 GDCPC(省赛) 和 SCNUCPC(校赛) 两算法比赛。
            \item 以下介绍可能会发生变化,请以最新官网消息为准。
        \end{itemize}
    \end{alertblock}
    \large{\textbf{\color[RGB]{37, 149, 68}打竞赛的好处}}
    \begin{enumerate}
        \item 面试时写在简历是加分项(特别是 ACM)。
        \item 综测加分,混奖学金和保研。
    \end{enumerate}
\end{frame}

\subsection{蓝桥杯}
\begin{frame}
    \textbf{\color[RGB]{34, 166, 242}蓝桥杯}分为省赛和国赛。省赛人均获奖,院内国赛每年大约10$\sim$20多人能进,平均国三。\\
    省赛和国赛一般都在每年上半年(即春季学期)。省赛三四月份,国赛六月份。\\
    赛制:OI 赛制。不可带资料的闭卷;且赛时无法得知代码是否通过;每道题通过部分测试点就得到部分得分。\\
    省赛和国赛各需要自费 300。几乎没人能拿到院奖金(上一届国二奖 700,国一奖 1500)。\\
    前 10\% 一等奖,20\% 二等,30\% 三等。即有效参赛里 60\% 得奖。\\
    题目难度逐年上升。\\
    比赛时长为 4h,通常是早上。\\
    有填空题(不限手段作答,甚至可用 Excel 解)和常规题。\\
    蓝桥杯是事实上不少先修班人的最终目的。\\
    与之对标的院内比赛为蓝桥杯热身赛。
\end{frame}

\subsection{天梯赛}
\begin{frame}
    \textbf{\color[RGB]{34, 166, 242}天梯赛}是国家级比赛。软院内会选拔约 30 人有资格参加。\\
    时间约是每年四月份。\\
    赛制:IOI 赛制。与 OI 赛制的唯一不同是赛时能看到代码对错与否。过一个测试点得对应分(分数分布不均)。\\
    有 100 分基础题(非算法送分题),100 分算法入门题和 90 分算法进阶题。其中个人奖总 175 分国三,总 220 分国二,总 250 国一。去年全校没有国二及以上。\\
    虽然写着是团队赛,但可粗暴理解为是个人赛并按个人去打就行,一般不用理会团队得分。\\
    题型都是传统题。\\
    只要能打到 100 分以上,不用花钱,学校报销。\\
    比赛时长是 3h,通常是下午。\\
    与之对标的院内比赛是天梯赛选拔赛。
\end{frame}

\subsection{CCF}
\begin{frame}
    \textbf{\color[RGB]{34, 166, 242}CCF认证}是考证,不算比赛。\\
    初赛每年有三场,平均一个季度一场。复赛(CSP)每年一场。下面仅介绍初赛。\\
    赛制:IOI 赛制。\\
    有五道题,每题 100 分。通常第一题非算法,第二题简单算法或非算法,第三题大模拟,第四第五题进阶算法。多数能打到 200$\sim$300 多分。\\
    100 分或以上不用花钱,学校报销。\\
    考试时长是 4h。
\end{frame}

\subsection{ACM}
\begin{frame}
    一般提到 \textbf{\color[RGB]{34, 166, 242}ACM} 即 \textbf{\color[RGB]{34, 166, 242}ICPC},全称是\textbf{国际大学生程序竞赛}。\\
    我们一般只能接触到区域赛,即(东)亚洲区域赛,实际上基本都在大陆打,参赛选手也基本是国人。\\
    我院的参赛前提是进入软院 ACM 集训队。一般每年 6 月份选拔,约选拔 20 人入队。\\
    区域赛的一般评奖规则是:设总人数为 $n$,则金牌人数为 $x=\min(0.1n, 35)$,银牌数约为 $2x$,铜牌数约为 $3x$。即前 210 名有奖。\\
    团队赛,三人一队参加。\\
    该比赛是业界最认可的 IT 行业竞赛,几乎没有之一。\\
    铜牌已经很厉害了(每年院内也就几支队伍拿铜);\\
    银牌企业面试大大加分(每年全校大约一两个银);\\
    金牌按理企业随便进(有史以来大概全校只有一个金)。
\end{frame}

\begin{frame}
    ACM 与上文比赛区别较大。首先题面全英,题目大约10到12题,题目排序与难度无关。比赛时间为 5h。\\
    赛制称为 ICPC 赛制,允许携带任何纸质资料。每道题只有全部测试点全对才能得分。除编译错误(CE)外通过(AC)前每次错误提交(WA,TLE,RE等)罚时 20 分钟。总用时是每道过了的题从比赛开始到首次通过的用时和加上罚时和。排名先按过题数后按用时和。\\
    题目难度较大,思维难度大,灵活,可能有的场过 2 题就能拿铜。\\
    三人一队一机。视疫情好坏可能举办线上或线下赛。线上一般人很多(六到八百队左右参赛),线下人比较少(大约两到四百队)。比赛费用全报销,线下可以去公费旅游\\(上文其他比赛都是在校内打的)。\\
    每年(届)每人最多正式参加两场区域赛。\\且学院名额一般很有限。\\
    比赛时长为 5h。
\end{frame}

\begin{frame}
    \textbf{\color[RGB]{34, 166, 242}CCPC} 是\textbf{中国大学生程序设计竞赛},其区域赛含金量约等于 ICPC 区域赛。赛制等基本与 ICPC 一致。有时用 XCPC 统称 ICPC 和 CCPC。\\
\end{frame}

\end{document}